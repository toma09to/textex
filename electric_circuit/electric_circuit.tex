\documentclass{ltjsreport}

\usepackage{amsmath}
\usepackage{siunitx}
\usepackage{tikz}
\usepackage{circuitikz}
\usepackage{bm}

\ctikzset{
    straight voltages,
    american currents,
    european resistors,
    american inductors
}

\newcounter{exercisecounter}[chapter]
\newenvironment{exercise}[1]{
    \refstepcounter{exercisecounter}
    \par\vspace{2mm}\noindent\textbf{演習\theexercisecounter}. 
}

\title{電気回路}
\author{toma09to}

\begin{document}

\chapter{電気回路とは}

電気回路とは、電気を利用するために電源、負荷などの回路素子を導体で接続したものである。
なお、「回路」の語義的には閉ループを指すが、アンテナなどの閉ループとなっていないものも慣例的に含めることがある。

\begin{figure}[htbp]
    \centering
    \begin{circuitikz}
        \draw (0,0) to[battery1,invert,v=$E$] (0,4) -- (2,4)
        to[R=$R$] (2,2)
        to[leDo] (2,0) -- (0,0);
    \end{circuitikz}
    \caption{電気回路の例}
    \label{電気回路の例}
\end{figure}

本書においては、このうち抵抗器・インダクタ・キャパシタ・電圧源・電流源のみを用いた回路(\emph{線形回路})を扱う。
線形回路の振る舞いは線形微分方程式で記述できる。

\chapter{基本的な物理量}

\section{電流}

導体中で電気を帯びた粒子(\emph{電荷})の集団が、連続的に移動する現象を\emph{電流}という。
導体のある断面における電流の大きさ$I$[\si{\A}]は、微小時間$dt$[\si{\s}]の間にその断面を通過した電荷の量が$dQ$[\si{\C}]であるとき、
\begin{equation}
    I = \frac{dQ}{dt}
\end{equation}
で表せる。

\begin{exercise}
    ?導線のある断面において\SI{8.0}{\s}間に\SI{20}{\C}の電荷が通過したとき、この断面の電流の大きさ$I$[\si{\A}]を求めよ。
    ただし、通過する電荷の量は常に一定であったとする。
\end{exercise}
\begin{exercise}
    ?$t$[\si{\s}]における電流の大きさが$2t$[\si{\A}]であったとき、\SI{0}{\s}から\SI{2.0}{\s}の間に導体の断面を通過した電荷の量$Q$[\si{\C}]を求めよ。
\end{exercise}

\section{電圧}

質点が万有引力を持つように、電荷はクーロン力を持つ。
ゆえに、電荷についても位置エネルギーを定義することができる。
これを\emph{電位}という。
そして、ある2点間の電位の差を\emph{電圧}(あるいは電位差)という。

ここで、電圧の大きさ$V$[\si{\V}]は、微小な電荷量$dQ$[\si{\C}]と、それがその電圧によってなされた仕事$dW$[\si{\J}]を用いて、
\begin{equation}
    V = \frac{dW}{dQ}
\end{equation}
と表せる。

なお、電流を生じさせる電圧を\emph{起電力}とよぶ。

\section{電力}

\emph{電力}とは、単位時間あたりに消費される電気エネルギーのことである。
エネルギーを消費する回路素子のことを、負荷と呼ぶ。
ある負荷の両端にかかる電圧を$V$[\si{\V}]、流れる電流を$I$[\si{\A}]とし、この負荷における電力を求めよう。

負荷において、微小時間$dt$[\si{s}]の間に消費されるエネルギーを$dW$[\si{J}]とすると電力$P$[\si{\W}]は、
\begin{equation}
    P = \frac{dW}{dt}
\end{equation}
となる。微分において連鎖律が成り立つから、以下のように変形できる。
\begin{equation}
    \frac{dW}{dt} = \frac{dQ}{dt}\frac{dW}{dQ} = VI
\end{equation}
したがって、電力は電圧と電流の積で表せる。

\section{電力量}

\emph{電力量}は、ある時間の間で消費されたエネルギーの合計のことである。
電力を$P$[\si{\W}]として、$t_1$[\si{\s}]から$t_2$[\si{\s}]の間に消費されたエネルギーすなわち電力量を$W$[\si{\W\s}]とすると、
\begin{equation}
    W = \int_{t_1}^{t_2} P dt
\end{equation}
となる。

電力量は仕事の一種であるから、単位は\si{\J}であるが、電力の単位が\si{\W}であるため\si{\W\s}が単位として用いられることが多い(\si{\J}と\si{\W\s}は等価)。

\chapter{回路素子の特性}

\section{抵抗器}

抵抗器における電流の流れにくさを、\emph{電気抵抗}といい、抵抗器の両端の電圧が$V$[\si{\V}]、流れる電流が$I$[\si{\A}]、電気抵抗が$R$[\si{\ohm}]のとき以下の法則が成り立つ。
\begin{equation}
    V = RI
\end{equation}
これを\emph{オームの法則}という。

また、電気抵抗の逆数$G (= R^{-1})$[\si{\siemens}]を\emph{コンダクタンス}といい、オームの法則より以下が成り立つ。
\begin{equation}
    I = GV
\end{equation}

抵抗器は電力を消費し、大抵の場合は熱エネルギーとして放出する。

\section{インダクタ}

インダクタとは、流れる電流によって形成される磁場にエネルギーを蓄えることができる回路素子のことである。

\end{document}
