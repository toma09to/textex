\documentclass[a4paper,12pt]{ltjsarticle}

\usepackage{amsmath}
\usepackage{amsfonts}
\usepackage{amssymb}
\usepackage{amsthm}
\usepackage{here}
\usepackage{mathtools}


\begin{document}
\section{記号の一覧}
\begin{table}[H]
    \centering
    \begin{tabular}{|c|c|c|}
        \hline
        記号 & 意味 & 例 \\
        \hline
        $\equiv, \coloneqq$ & 定義 & $A \equiv \{1, 2, 3\}$ \\
        \hline
        $\in$ & 集合に属する & $3 \in A$ \\
        \hline
        $\notin$ & 集合に属さない & $-1 \notin A$ \\
        \hline
        $\underset{\text{def}}{\Longleftrightarrow}$ & 定義 & $a > b \underset{\text{def}}{\Longleftrightarrow} aはbより大きい$ \\
        \hline
        $\subseteq$ & 部分集合 & $\{ 1, 3 \} \subseteq \{ 1, 2, 3 \}$ \\
        \hline
        $\forall$ & 任意の & $\forall x \in \mathbb{R} ; x^2 \geq 0$ \\
        \hline
        $\exists$ & 存在する & $\exists n \in \mathbb{N} ; x^2 = 4$ \\
        \hline
        $\mathcal{P}(\bullet)$ & べき集合 & $\mathcal{P}(\{ a, b \}) = \{ \{ \}, \{ a \}, \{ b \}, \{ a, b \} \}$ \\
        \hline
        $\mid \bullet \mid$ & 濃度 & $\mid \{ 0, 1, 2, 3, 4 \} \mid = 5$ \\
        \hline
        $\cup$ & 和集合 & $\{ 1, 2 \} \cup \{ 2, 3 \} = \{ 1, 2, 3 \}$ \\
        \hline
        $\cap$ & 積集合 & $\{ 1, 2 \} \cap \{ 2, 3 \} = \{ 2 \}$ \\
        \hline
        $-$ & 差集合 & $\{ 1, 2 \} - \{ 2, 3 \} = \{ 1 \}$ \\
        \hline
        $\times$ & 直積 & $\{ 1, 2 \} \times \{ 2, 3 \} = \{ (1,2), (1,3), (2,2), (2,3) \}$ \\
        \hline
        $(\bullet, \bullet)$ & 順序対 & $(a,b) \neq (b,a)$ \\
        \hline
        $\bar{\bullet}$ & 補集合 & $A - B = A \cap \bar{B}$ \\
        \hline
        $〜$ & 否定 & $〜S \Leftrightarrow 「Sでない」$ \\
        \hline
        $\lor$ & 論理和 & $S \lor T \Leftrightarrow 「SまたはT」$ \\
        \hline
        $\land$ & 論理積 & $S \land T \Leftrightarrow 「SかつT」$ \\
        \hline
        $\Rightarrow$ & 含意 & $S \Rightarrow T \Leftrightarrow 「SならばT」$ \\
        \hline
        $\mid$ & 割り切る & $4 \mid n \Leftrightarrow n = 4k(k \in \mathbb{Z})$ \\
        \hline
    \end{tabular}
\end{table}

\section{記号の定義}
\begin{table}[H]
    \centering
    \begin{tabular}{|c|c|}
        \hline
        記号 & 定義 \\
        \hline
        $A \subseteq B$ & $\forall a \in A ; a \in B$ \quad Aのすべての要素がBに属する \\
        \hline
        $A \cup B$ & $\{ x \mid x \in A \lor x \in B \}$ \\
        \hline
        $A \cap B$ & $\{ x \mid x \in A \land x \in B \}$ \\
        \hline
        $A - B$ & $\{ x \mid x \in A \land x \notin B \}$ \\
        \hline
        $A \times B$ & $\{ (a,b) \mid a \in A \land b \in B \}$ \\
        \hline
    \end{tabular}
\end{table}

\section{特定の集合}
\begin{table}[H]
    \centering
    \begin{tabular}{|c|c|}
        \hline
        記号 & 意味 \\
        \hline
        $\varnothing$ & 空集合 \\
        \hline
        $\mathbb{N}$ & 自然数全体の集合 \\
        \hline
        $\mathbb{Z}$ & 整数全体の集合 \\
        \hline
        $\mathbb{Q}$ & 有理数全体の集合 \\
        \hline
        $\mathbb{R}$ & 実数全体の集合 \\
        \hline
        $\mathbb{C}$ & 複素数全体の集合 \\
        \hline
    \end{tabular}
\end{table}
\end{document}
