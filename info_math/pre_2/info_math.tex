\documentclass[a4paper,12pt]{ltjsarticle}

\usepackage{amsmath}
\usepackage{amsfonts}
\usepackage{amssymb}
\usepackage{amsthm}
\usepackage{enumerate}
\usepackage{mathtools}

\title{情報数学前期期末試験直前演習問題}
\author{toma09to}
\date{}

\newcommand{\N}{\mathbb{N}}
\newcommand{\Z}{\mathbb{Z}}
\newcommand{\Q}{\mathbb{Q}}
\newcommand{\R}{\mathbb{R}}

\begin{document}
\setcounter{section}{-1}
\maketitle

間違い等ありましたら,連絡してください.

本資料を参考にしたことによって生じた損害について,当方は一切の責任を負いません.

\section{連続する2つの整数の積は偶数である.}
\begin{proof}
    2整数の小さい方を$n$とする. \\
    $2 \mid n$のとき,$n = 2k$として,
    \begin{align*}
        n(n+1) &= n^2 + n \\
               &= (2k)^2 + 2k \\
               &= 4k^2 + 2k \\
               &= 2(2k^2 + k)
    \end{align*}
    $2 \nmid n$のとき,$n = 2k + 1$として,
    \begin{align*}
        n(n+1) &= n^2 + n \\
               &= (2k + 1)^2 + (2k + 1) \\
               &= 4k^2 + 6k + 2 \\
               &= 2(2k^2 + 3k + 1)
    \end{align*}
\end{proof}

\section{$n \in \Z$のとき,$n^3 - n$は偶数であることを証明せよ.}
\begin{proof}
    $2 \mid n$のとき,$n = 2k$として,
    \begin{align*}
        n^3 - n &= n(n + 1)(n - 1) \\
                &= 2k(2k + 1)(2k - 1)
    \end{align*}
    $2 \nmid n$のとき,$n = 2k + 1$として,
    \begin{align*}
        n^3 - n &= n(n + 1)(n - 1) \\
                &= 2k(2k + 1)(2k + 2)
    \end{align*}
\end{proof}

\section{$x,y \in \Z$とする.$xy$および$x + y$が偶数であるとき,$x$と$y$の両方とも偶数であることを証明せよ.}
\begin{proof}
    $xy = 2k$,$x + y = 2l$とすると,$y = 2l - x$ \\
    これを$xy$に代入して,
    \begin{align*}
        xy &= x(2l - x) \\
           &= 2lx - x^2 \\
        \therefore x^2 &= 2lx - 2k = 2(lx - k)
    \end{align*}
    ここで,$2 \mid x^2 \Rightarrow 2 \mid x$を証明する.\\
    この対偶は,$2 \nmid x \Rightarrow 2 \nmid x^2$ \\
    $x = 2m + 1$として,
    \begin{equation*}
        x^2 = (2m + 1)^2 = 4m^2 + 4m + 1 = 2(2m^2 + 2m) + 1
    \end{equation*}
    したがって,$x$は偶数で,$x = 2n$とおくと,
    \begin{equation*}
        y = 2l - x = 2l - 2n = 2(l - n)
    \end{equation*}
    よって,$y$も偶数.
\end{proof}

\section{整数$a$,$b$,$c$について,$a + b, b + c, c + a$の少なくとも一つは偶数であることを示せ.}
\begin{proof}
    $a + b$,$b + c$,$c + a$が全て奇数であると仮定し,
    \begin{align*}
        a + b &= 2k + 1 \\
        b + c &= 2l + 1 \\
        c + a &= 2m + 1
    \end{align*}
    とおく.
    このとき,
    \begin{align*}
        (a + b) + (b + c) &= (2k + 1) + (2l + 1) \\
        a + 2b + c &= 2k + 2l + 2 \\
        c + a &= 2k + 2l - 2b + 2 = 2(k + l - b + 1)
    \end{align*}
    これは,$c + a$が奇数であることと矛盾.
\end{proof}

\section{$x$を正の実数とする.$x - \frac{2}{x} > 1$のとき$x > 2$であることを(a)直接証明(b)対偶による証明(c)背理法による証明により証明せよ.}
\begin{enumerate}[(a)]
    \item
        \begin{proof}[直接証明]
            \begin{equation*}
                x - \frac{2}{x} > 1
            \end{equation*}
            両辺にxを掛けて,
            \begin{gather*}
                x^2 - 2 > x \\
                x^2 - x - 2 > 0 \\
                (x - 2)(x + 1) > 0
            \end{gather*}
            両辺をx + 1で割って,
            \begin{gather*}
                x - 2 > 0 \\
                x > 2
            \end{gather*}
        \end{proof}
    \item
        \begin{proof}[対偶による証明]
            対偶は,「$x \leq 2 \Rightarrow x - \frac{2}{x} \leq 1$」 \\
            $2 \geq x$の両辺を$-x$で割って,
            \begin{equation*}
                -\frac{2}{x} \leq -1
            \end{equation*}
            $x \leq 2$であるから両辺に$x$を足して,
            \begin{equation*}
                x - \frac{2}{x} \leq x - 1 \leq 2 - 1 = 1
            \end{equation*}
        \end{proof}
    \item
        \begin{proof}[背理法による証明]
            $x - \frac{2}{x} > 1$かつ$x \leq 2$と仮定する. \\
            $x \leq 2$より(b)と同様の変形をして,
            \begin{equation*}
                x - \frac{2}{x} \leq 1
            \end{equation*}
            を得る.これは,$x - \frac{2}{x} > 1$と矛盾.
        \end{proof}
\end{enumerate}

\section{次の命題を証明せよ.$n,m \in \Z$とする.}
\begin{enumerate}[(1)]
    \item $3 \mid m$ならば$3 \mid m^2$
        \begin{proof}
            $m = 3k$として,
            \begin{equation*}
                m^2 = (3k)^2 = 9k^2 = 3(3k^2)
            \end{equation*}
        \end{proof}
    \item $3 \mid m^2$ならば$3 \mid m$
        \begin{proof}
            対偶は,「$3 \nmid m \Rightarrow 3 \nmid m^2$」 \\
            \begin{enumerate}[(I)]
                \item $m = 3k + 1$のとき
                    \begin{equation*}
                        m^2 = (3k + 1)^2 = 9k^2 + 6k + 1 = 3(3k^2 + 2k) + 1
                    \end{equation*}
                \item $m = 3k + 2$のとき
                    \begin{equation*}
                        m^2 = (3k + 2)^2 = 9k^2 + 12k + 4 = 3(3k^2 + 4k + 1) + 1
                    \end{equation*}
            \end{enumerate}
        \end{proof}
    \item $3n^2 \mid m^2$ならば$3 \mid m$
        \begin{proof}
            $3n^2 \mid m^2$より,$3 \mid m^2$ \\
            (2)より,$3 \mid m^2 \Rightarrow 3 \mid m$
        \end{proof}
    \item $3n^2 = m^2$ならば$3 \mid n$
        \begin{proof}
            $3n^2 = m^2$より,$3 \mid m^2$ \\
            (2)より,$3 \mid m$であるから,$m = 3k$とおいて,
            \begin{align*}
                (3k)^2 &= 3n^2 \\
                3k^2 &= n^2
            \end{align*}
            したがって,$3 \mid n^2$すなわち$3 \mid n$
        \end{proof}
    \item $3n^2 = m^2$ならば$n$と$m$は1以外の公約数を持つ.
        \begin{proof}
            (2),(4)より,$3 \mid m$かつ$3 \mid n$ \\
            よって,$n$と$m$は3を公約数に持つ.
        \end{proof}
    \item $\sqrt{3}$は無理数である.
        \begin{proof}
            $\sqrt{3}$が有理数であると仮定する. \\
            すると,$p,q \in \Z$を用いて,
            \begin{equation*}
                \sqrt{3} = \frac{q}{p} \quad (\frac{q}{p}は既約分数)
            \end{equation*}
            両辺を2乗して変形すると,
            \begin{equation*}
                3q^2 = p^2
            \end{equation*}
            (5)より,$p$と$q$は1以外の公約数を持つ.これは,$\frac{q}{p}$が既約分数であることと矛盾.
        \end{proof}
\end{enumerate}

\section{以下の問題に答えよ.}
\begin{enumerate}[(1)]
    \item $a$を実数とする.次の\square に適切な記号を一文字入れよ(証明不要). \\
        逆数をとっても符号は変わらない.
        \begin{align*}
            a > 0 &\Leftrightarrow \frac{1}{a} > 0 \\
            a < 0 &\Leftrightarrow \frac{1}{a} < 0
        \end{align*}
    \item $a,b$を実数とする.次の\square に適切な文字列を入れよ(証明不要). \\
        $ab > 0$なのは,$a > 0 \land b > 0$または$a < 0 \land b < 0$のときである.
        \begin{equation*}
            ab > 0 \Leftrightarrow a と b の符号は同じ
        \end{equation*}
    \item 上の結果を用いて「$\frac{1}{r(1 - r)}が4以上の実数 \Rightarrow 0 < r < 1$」を示せ.
        \begin{proof}
            \begin{equation*}
                \frac{1}{r(1 - r)} = \frac{1}{r} \times \frac{1}{1 - r} > 0
            \end{equation*}
            となるのは,$\frac{1}{r}$と$\frac{1}{1 - r}$の符号すなわち$r$と$1 - r$の符号が同じとき.
            \begin{enumerate}[(I)]
                \item $r > 0$かつ$1 - r > 0$のとき \\
                    これらを同時に満たすのは$0 < r < 1$のとき.
                \item $r < 0$かつ$1 - r < 0$のとき \\
                    これらを同時に満たす$r$は存在しない.
            \end{enumerate}
            よって
            \begin{equation*}
                \frac{1}{r(1 - r)} > 0 \Rightarrow 0 < r < 1
            \end{equation*}
            であるから,
            \begin{equation*}
                \frac{1}{r(1 - r)} \geq 4 \Rightarrow 0 < r < 1
            \end{equation*}
            も成り立つ.
        \end{proof}
\end{enumerate}

\section{整数を(正の)整数で割ると,商と余りが必ず一つずつ存在する.これを前提として以下の問いに答えよ.}
\begin{enumerate}[(1)]
    \item 0でない整数$a$と$b$に対して,
        \begin{equation*}
            I_{a,b} \coloneqq \{ ax + by \mid x,y \in \Z \}
        \end{equation*}
        とおくことにすると
        \begin{gather*}
            w \in \Z, z \in I_{a,b} \Rightarrow wz \in I_{a,b}, \\
            z_1, z_2 \in I_{a,b} \Rightarrow z_1 + z_2 \in I_{a,b}
        \end{gather*}
        が成り立つことを示せ.
        \begin{proof}
            $z = ax + by$とおくと,
            \begin{equation*}
                wz = w(ax + by) = a(wx) + b(wy) \in I_{a,b}
            \end{equation*}
            また,$z_1 = ax_1 + by_1, z_2 = ax_2 + by_2$とおくと,
            \begin{equation*}
                z_1 + z_2 = ax_1 + by_1 + ay_2 + by_2 = a(x_1 + x_2) + b(y_1 + y_2) \in I_{a,b}
            \end{equation*}
        \end{proof}
    \item $\{ z \in I_{a,b} \mid z > 0 \}$を考える.この集合が空集合でないことを示せ.
        \begin{proof}
            $z = ax + by$とおいて,$x = a, y = b$のとき
            \begin{equation*}
                z = a \times a + b \times b = a^2 + b^2 > 0
            \end{equation*}
            したがって,要素が1つ以上存在するため空集合ではない.
        \end{proof}
    \item $d \coloneqq min\{ z \in I_{a,b} \mid z > 0 \}$とおく.以下の命題を示せ.
        \begin{enumerate}
            \item[(3-1)] $c \in I_{a,b}$として,$r = c \% d$とすると$r \in I_{a,b}$となる.
                \begin{proof}
                    $c = ax_c + by_c$,$d = ax_d + by_d$とおく. \\
                    $r = c \% d$より,$n \in \Z$を用いて
                    \begin{align*}
                        c &= nd + r \\
                        r &= c - nd \\
                          &= (ax_c + by_c) - n(ax_d + by_d) \\
                          &= a(x_c - nx_d) + b(y_c - ny_d)
                    \end{align*}
                    よって,$r \in I_{a,b}$
                \end{proof}
            \item[(3-2)] $d$は$a$と$b$の公約数である.
                \begin{proof}
                    $r = c \% d$より,$0 \leq r < d$ \\
                    ここで,$r > 0$であると仮定すると,(3-1)より$r \in I_{a,b}$ \\
                    しかし,これは$d = min\{ z \in I_{a,b} \mid z > 0 \}$と矛盾. \\
                    よって$r = 0$.ゆえに$n \in \Z$を用いて
                    \begin{equation*}
                        c = nd
                    \end{equation*}
                    ここで,
                    \begin{gather*}
                        a = a \times 1 + b \times 0 \\
                        b = a \times 0 + b \times 1
                    \end{gather*}
                    より$a,b \in I_{a,b}$.これより$n,m \in \Z$を用いて
                    \begin{gather*}
                        a = nd \\
                        b = md
                    \end{gather*}
                    すなわち$d \mid a$,$d \mid b$. \\
                    以上より,$d$は$a$と$b$の公約数である.
                \end{proof}
        \end{enumerate}
\end{enumerate}

\section{$n:整数$とする.次を示せ.}
\begin{enumerate}[(1)]
    \item $7n + 4 : 偶数 \Leftrightarrow 3n - 11 : 奇数$
        \begin{proof}
            \quad \\
            \begin{itemize}
                \item $2 \mid (7n + 4) \Rightarrow 2 \nmid (3n - 11)$ \\
                    $7n + 4 = 2k$とおくと
                    \begin{equation*}
                        3n - 11 = 2k - 4n - 15 = 2(k - 2n - 8) + 1
                    \end{equation*}
                \item $2 \mid (7n + 4) \Leftarrow 2 \nmid (3n - 11)$ \\
                    $3n - 11 = 2l + 1$とおくと
                    \begin{equation*}
                        7n + 4 = 2l + 4n + 16 = 2(l + 2n + 8)
                    \end{equation*}
            \end{itemize}
            以上より$2 \mid (7n + 4) \Leftrightarrow 2 \nmid (3n - 11)$
        \end{proof}
    \item $5n^2 : 偶数 \Leftrightarrow 3n^3 : 偶数$
        \begin{proof}
            \quad \\
            \begin{itemize}
                \item $2 \mid 5n^2 \Rightarrow 2 \mid 3n^3$ \\
                    $5n^2 = 2k$とおくと
                    \begin{equation*}
                        3n^3 = n(5n^2 - 2n^2) = n(2k - 2n^2) = 2n(k - n^2)
                    \end{equation*}
                \item $2 \mid 5n^2 \Leftarrow 2 \mid 3n^3$ \\
                    $2 \mid 3n^3 \Rightarrow 2 \mid n$を証明する. \\
                    対偶は「$2 \nmid n \Rightarrow 2 \nmid 3n^3$」.$n = 2l + 1$とおいて
                    \begin{equation*}
                        3n^3 = 3(2l + 1)^3 = 3(8l^3 + 12l^2 + 6l + 1) = 2(12l^3 + 18l^2 + 9l + 1) + 1
                    \end{equation*}
                    次に,$2 \mid n \Rightarrow 2 \mid 5n^2$を証明する. \\
                    $n = 2m$とおいて
                    \begin{equation*}
                        5n^2 = 5(2m)^2 = 2(10m^2)
                    \end{equation*}
            \end{itemize}
            以上より$2 \mid 5n^2 \Leftrightarrow 2 \mid 3n^3$
        \end{proof}
    \item $n^2 - 3n  + 1$は奇数
        \begin{proof}
            $2 \mid n$のとき$n = 2k$とおいて
            \begin{equation*}
                n^2 - 3n + 1 = (2k)^2 - 3(2k) + 1 = 4k^2 - 6k + 1 = 2(2k^2 - 3k) + 1
            \end{equation*}
            $2 \nmid n$のとき$n = 2k + 1$とおいて
            \begin{equation*}
                n^2 - 3n + 1 = (2k + 1)^2 - 3(2k + 1) + 1 = 4k^2 - 2k - 1 = 2(2k^2 - k - 1) + 1
            \end{equation*}
        \end{proof}
    \item $n + m : 偶数 \Leftrightarrow nとmは同パリティを持つ$
        \begin{proof}
            \quad \\
            \begin{itemize}
                \item $2 \mid (n + m) \Rightarrow nとmは同パリティを持つ$ \\
                    対偶は「$nとmは逆パリティを持つ \Rightarrow 2 \nmid (n + m)$」
                    \begin{enumerate}[(I)]
                        \item $2 \mid n, 2 \nmid m$のとき$n = 2k, m = 2l + 1$とおいて
                            \begin{equation*}
                                n + m = 2k + 2l + 1 = 2(k + l) + 1
                            \end{equation*}
                        \item $2 \nmid n, 2 \mid m$のとき$n = 2k + 1, m = 2l$とおいて
                            \begin{equation*}
                                n + m = 2k + 2l + 1 = 2(k + l) + 1
                            \end{equation*}
                    \end{enumerate}
                \item $2 \mid (n + m) \Leftarrow nとmは同パリティを持つ$ \\
                    \begin{enumerate}[(I)]
                        \item $2 \mid n, 2 \mid m$のとき$n = 2k, m = 2l$とおいて
                            \begin{equation*}
                                n + m = 2k + 2l = 2(k + l)
                            \end{equation*}
                        \item $2 \nmid n, 2 \nmid m$のとき$n = 2k + 1, m = 2l + 1$とおいて
                            \begin{equation*}
                                n + m = 2k + 2l + 2 = 2(k + l + 1)
                            \end{equation*}
                    \end{enumerate}
            \end{itemize}
            以上より$2 \mid (n + m) \Leftrightarrow nとmは同パリティを持つ$
        \end{proof}
    \item $3n + 5m : 偶数 \Leftrightarrow nとmは同パリティを持つ$
        \begin{proof}
            \quad \\
            \begin{itemize}
                \item $2 \mid (3n + 5m) \Rightarrow nとmは同パリティを持つ$ \\
                    対偶は「$nとmは逆パリティを持つ \Rightarrow 2 \nmid (3n + 5m)$」
                    \begin{enumerate}[(I)]
                        \item $2 \mid n, 2 \nmid m$のとき$n = 2k, m = 2l + 1$とおいて
                            \begin{equation*}
                                3n + 5m = 3(2k) + 5(2l + 1) = 6k + 10l + 5 = 2(3k + 5l + 2) + 1
                            \end{equation*}
                        \item $2 \nmid n, 2 \mid m$のとき$n = 2k + 1, m = 2l$とおいて
                            \begin{equation*}
                                3n + 5m = 3(2k + 1) + 5(2l) = 6k + 10l + 3 = 2(3k + 5l + 1) + 1
                            \end{equation*}
                    \end{enumerate}
                \item $2 \mid (3n + 5m) \Leftarrow nとmは同パリティを持つ$ \\
                    \begin{enumerate}[(I)]
                        \item $2 \mid n, 2 \mid m$のとき$n = 2k, m = 2l$とおいて
                            \begin{equation*}
                                3n + 5m = 3(2k) + 5(2l) = 6k + 10l = 2(3k + 5l)
                            \end{equation*}
                        \item $2 \nmid n, 2 \nmid m$のとき$n = 2k + 1, m = 2l + 1$とおいて
                            \begin{equation*}
                                3n + 5m = 3(2k + 1) + 5(2l + 1) = 6k + 10l + 8 = 2(3k + 5l + 4)
                            \end{equation*}
                    \end{enumerate}
            \end{itemize}
            以上より$2 \mid (3n + 5m) \Leftrightarrow nとmは同パリティを持つ$
        \end{proof}
    \item $3n + 1$と$5n + 2$は逆パリティを持つ
        \begin{proof}
            \begin{equation*}
                (3n + 1) + (5n + 2) = 8n + 3 = 2(4n + 1) + 1
            \end{equation*}
            (4)より,$2 \nmid (n + m) \Leftrightarrow nとmは逆パリティを持つ$が成り立つ. \\
            したがって,$3n + 1$と$5n + 2$は逆パリティを持つ.
        \end{proof}
\end{enumerate}

\section{次の式を示せ.但し,$\Z$はすべての整数からなる集合,$\R$はすべての実数からなる集合とする.}
\begin{enumerate}[(1)]
    \item $\{ 3n \mid n \in \Z \} = \{ 6m + 9n \mid n,m \in \Z \}$
        \begin{proof}
            \quad \\
            \begin{itemize}
                \item $\{ 3n \mid n \in \Z \} \subseteq \{ 6m + 9n \mid n,m \in \Z \}$
                    \begin{equation}
                        3n = -6n + 9n = 6(-n) + 9n
                    \end{equation}
                \item $\{ 3n \mid n \in \Z \} \supseteq \{ 6m + 9n \mid n,m \in \Z \}$
                    \begin{equation}
                        6m + 9n = 3(2m + 3n)
                    \end{equation}
            \end{itemize}
            以上より$\{ 3n \mid n \in \Z \} = \{ 6m + 9n \mid n,m \in \Z \}$
        \end{proof}
    \item $\{ x \in \R \mid x^2 - 3x + 2 \leq 0 \} = \{ x \in \R \mid 1 \leq x \leq 2 \}$
        \begin{proof}
            \quad \\
            \begin{itemize}
                \item $\{ x \in \R \mid x^2 - 3x + 2 \leq 0 \} \subseteq \{ x \in \R \mid 1 \leq x \leq 2 \}$
                    \begin{align*}
                        x^2 - 3x + 2 &\leq 0 \\
                        (x - 1)(x - 2) &\leq 0 \\
                        \therefore 1 \leq x &\leq 2
                    \end{align*}
                \item $\{ x \in \R \mid x^2 - 3x + 2 \leq 0 \} \supseteq \{ x \in \R \mid 1 \leq x \leq 2 \}$ \\
                    $x \geq 1, x \leq 2$より$x - 1 \geq 0, x - 2 \leq 0$
                    \begin{align*}
                        \therefore (x - 1)(x - 2) &\leq 0 \\
                        x^2 - 3x + 2 &\leq 0
                    \end{align*}
            \end{itemize}
        \end{proof}
    \item $\{ n \in \Z \mid n \geq 20 \} \subseteq \{ 5n + 6m | n,m \in \Z, n \geq 0, m \geq 0 \}$
        \begin{proof}
            0以上の整数$k$を用いて,左の集合の要素$n$は$n = 20 + k$と表される.
            \begin{enumerate}[(I)]
                \item $k = 5l$のとき
                    \begin{equation*}
                        n = 20 + 5l = 5 \times (l + 4) + 6 \times 0
                    \end{equation*}
                \item $k = 5l + 1$のとき
                    \begin{equation*}
                        n = 21 + 5l = 5 \times (l + 3) + 6 \times 1
                    \end{equation*}
                \item $k = 5l + 2$のとき
                    \begin{equation*}
                        n = 22 + 5l = 5 \times (l + 2) + 6 \times 2
                    \end{equation*}
                \item $k = 5l + 3$のとき
                    \begin{equation*}
                        n = 23 + 5l = 5 \times (l + 1) + 6 \times 3
                    \end{equation*}
                \item $k = 5l + 4$のとき
                    \begin{equation*}
                        n = 24 + 5l = 5 \times l + 6 \times 4
                    \end{equation*}
            \end{enumerate}
            よって$\{ n \in \Z \mid n \geq 20 \} \subseteq \{ 5n + 6m | n,m \in \Z, n \geq 0, m \geq 0 \}$
        \end{proof}
\end{enumerate}

\section{背理法を用いて次を示せ}
\begin{enumerate}[(1)]
    \item $nが0より大きな整数 \Rightarrow 2n < n^2 < 3n$とはならない.
        \begin{proof}
            $n > 0$かつ$2n < n^2 < 3n$とする. \\
            $n > 0$であるから,各辺を$n$で割って,
            \begin{equation*}
                2 < n < 3
            \end{equation*}
            これを満たす整数$n$は存在しない.
        \end{proof}
    \item $\{ x \in \R \mid x \neq 0, x + \frac{1}{x} < 2 \} = \{ x \in \R \mid x < 0\}$
        \begin{proof}
            \quad \\
            \begin{itemize}
                \item $\{ x \in \R \mid x \neq 0, x + \frac{1}{x} < 2 \} \subseteq \{ x \in \R \mid x < 0\}$ \\
                    $x + \frac{1}{x} < 2$かつ$x \geq 0$と仮定する. \\
                    $x + \frac{1}{x} < 2$の両辺に$x$を掛けて
                    \begin{gather*}
                        x^2 + 1 < 2x \\
                        x^2 - 2x + 1 < 0 \\
                        (x - 1)^2 < 0
                    \end{gather*}
                    これを満たす$x$は存在せず,$x \geq 0$と矛盾.
                \item $\{ x \in \R \mid x \neq 0, x + \frac{1}{x} < 2 \} \supseteq \{ x \in \R \mid x < 0\}$
                    $x < 0$かつ$x + \frac{1}{x} \geq 2$と仮定する. \\
                    $x < 0$の両辺に$(x - 1)^2$を掛けて
                    \begin{gather*}
                        x(x - 1)^2 < 0 \\
                        x(x^2 - 2x + 1) < 0 \\
                        x^3 - 2x^2 + x < 0 \\
                        x^3 + x < 2x^2
                    \end{gather*}
                    両辺を$x^2$で割って
                    \begin{equation*}
                        x + \frac{1}{x} < 2
                    \end{equation*}
                    これは$x + \frac{1}{x} \geq 2$と矛盾.
            \end{itemize}
        \end{proof}
    \item $\{ n \in \Z \mid 5n^2は偶数 \} = \{ n \in \Z \mid 3n^3は偶数 \}$
        \begin{proof}
            \quad \\
            \begin{itemize}
                \item $n \in (左) \Rightarrow n \in (右)$を示す.
                    $2 \mid 5n^2 \Rightarrow 2 \mid 3n^3$を証明する. \\
                    $5n^2 = 2k$とおくと
                    \begin{equation*}
                        3n^3 = (5n^2 - 2n^2)n = (2k - 2n^2)n = 2(k - n^2)n
                    \end{equation*}
                    よって$2 \mid 5n^2 \Rightarrow 2 \mid 3n^3$ \\
                    すなわち$\{ n \in \Z \mid 5n^2は偶数 \} \subseteq \{ n \in \Z \mid 3n^3は偶数 \}$
                \item $n \in (左) \Leftarrow n \in (左)$を示す.
                    まず,$2 \mid 3n^3 \Rightarrow 2 \mid n$を証明する. \\
                    $2 \mid 3n^3 \land 2 \nmid n$と仮定し,$n = 2k + 1$とすると
                    \begin{equation*}
                        3n^3 = 3(2k + 1)^3 = 2(12k^3 + 18k^2 + 9k + 1) + 1
                    \end{equation*}
                    これは$2 \mid 3n^3$に矛盾.よって$2 \mid 3n^3 \Rightarrow 2 \mid n$ \\
                    次に,$2 \mid n \Rightarrow 2 \mid 5n^2$を証明する. \\
                    $n = 2l$とおくと
                    \begin{equation*}
                        5n^2 = 5(2l)^2 = 2(10l^2)
                    \end{equation*}
                    よって$2 \mid n \Rightarrow 2 \mid 5n^2$.したがって,$2 \mid 3n^3 \Rightarrow 2 \mid 5n^2$ \\
                    すなわち$\{ n \in \Z \mid 5n^2は偶数 \} \supseteq \{ n \in \Z \mid 3n^3は偶数 \}$
            \end{itemize}
            したがって,$\{ n \in \Z \mid 5n^2は偶数 \} = \{ n \in \Z \mid 3n^3は偶数 \}$
        \end{proof}
    \item $\{ (n,m) \mid n^2 = 2m^2 \} \subseteq \{ (n,m) \mid nとmは偶数 \}$
        \begin{proof}
            $A = \{ (n,m) \mid n^2 = 2m^2 \}, B = \{ (n,m) \mid nとmは偶数 \}$とする. \\
            $(n,m) \in A \land (n,m) \notin B$を満たす$(n,m)$が存在すると仮定する. \\
            $n^2 = 2m^2$より$2 \mid n^2$ \\
            $2 \mid n^2 \Rightarrow 2 \mid n$(証明略)であるから,$2 \mid n$ \\
            $n = 2k$とおいて
            \begin{align*}
                (2k)^2 &= 2m^2 \\
                2k^2 &= m^2
            \end{align*}
            同様に$2 \mid m$が成り立ち,$n$と$m$が偶数であるから$(n,m) \in B$ \\
            これは$(n,m) \notin B$と矛盾.
        \end{proof}
\end{enumerate}

\section{集合$A, B, X$に対して以下の式を示せ.}
\begin{equation*}
    (A \cap (B \cup X)) \cup (B \cap X) = (A \cap B) \cup (B \cap X) \cup (X \cap A)
\end{equation*}
※記述がだいぶ怪しいです.出ないことを祈りましょう.
\begin{proof}
    $A \cap (B \cup X) = (A \cap B) \cup (X \cap A)$を示す.
    \begin{itemize}
        \item $A \cap (B \cup X) \subseteq (A \cap B) \cup (X \cap A)$ \\
            $x \in A \cap (B \cup X)$とすると \\
            積集合の定義より$x \in A$かつ$x \in B \cup X$ \\
            和集合の定義より$x \in B$または$x \in X$であるから \\
            $x \in A$かつ$x \in B$すなわち$x \in A \cap B$ \\
            または$x \in A$かつ$x \in X$すなわち$x \in X \cap A$が成り立ち, \\
            和集合の定義より$x \in (A \cap B) \cup (X \cap A)$となる.
        \item $A \cap (B \cup X) \supseteq (A \cap B) \cup (X \cap A)$ \\
            $x \in (A \cap B) \cup (X \cap A)$とすると \\
            和集合の定義より$x \in A \cap B$または$x \in X \cap A$ \\
            積集合の定義より$x \in A$かつ$x \in B$ または $x \in A$かつ$x \in X$である. \\
            したがって,$x \in A$かつ$x \in B \cup X$が成り立ち, \\
            積集合の定義より$x \in A \cap (B \cup X)$となる.
    \end{itemize}
    これを用いて命題を証明する.
    \begin{itemize}
        \item $(A \cap (B \cup X)) \cup (B \cap X) \subseteq (A \cap B) \cup (B \cap X) \cup (X \cap A)$ \\
            $x \in (A \cap (B \cup X)) \cup (B \cap X)$とすると \\
            $x \in ((A \cap B) \cup (X \cap A)) \cup (B \cap X)$ \\
            すなわち$x \in A \cap B$または$x \in B \cap X$または$x \in X \cap A$が成り立つ. \\
            したがって,$x \in (A \cap B) \cup (B \cap X) \cup (X \cap A)$
        \item $(A \cap (B \cup X)) \cup (B \cap X) \supseteq (A \cap B) \cup (B \cap X) \cup (X \cap A)$ \\
            $x \in (A \cap B) \cup (B \cap X) \cup (X \cap A)$とすると \\
            $x \in A \cap B$または$x \in B \cap X$または$x \in X \cap A$が成り立つ. \\
            したがって,$x \in A \cap (B \cup X)$または$x \in B \cap X$ \\
            すなわち$x \in (A \cap (B \cup X)) \cup (B \cap X)$が成り立つ.
    \end{itemize}
\end{proof}

\section{$a,b \in \Z$とする.$a^2 + 2b^2 \equiv 0 \pmod 3$のとき$a$および$b$はともに3を法として0に合同であるか,3を法としてともに0に合同でないかのいずれかであることを証明せよ.}
以下,合同式は全て3を法とするものとする.
\begin{proof}
    対偶は,「$(a \not\equiv 0 \land b \equiv 0) \lor (a \equiv 0 \land b \not\equiv 0) \Rightarrow a^2 + 2b^2 \not\equiv 0$」
    \begin{enumerate}[(I)]
        \item $a \not\equiv 0 \land b \equiv 0$のとき \\
            $a = 3k + l(l = 1,2), b = 3m$とおくと
            \begin{equation*}
                a^2 + 2b^2 = (3k + l)^2 + 2(3m)^2 = 3(3k^2 + 2kl + 6m^2) + l^2
            \end{equation*}
            $1^2 \not\equiv 0 , 2^2 \not\equiv 0$であるから,$a^2 + 2b^2 \not\equiv 0$
        \item $a \equiv 0 \land b \not\equiv 0$のとき \\
            $a = 3m, b = 3k + l(l = 1,2)$とおくと
            \begin{equation*}
                a^2 + 2b^2 = (3m)^2 + 2(3k + l)^2 = 3(6k^2 + 4kl + 3m^2) + 2l^2
            \end{equation*}
            $2 \times 1^2 \not\equiv 0 , 2 \times 2^2 \not\equiv 0$であるから,$a^2 + 2b^2 \not\equiv 0$
    \end{enumerate}
\end{proof}

\section{3つの整数$a, b, c$が$a^2 + b^2 = c^2$を満たすとき,$a, b, c$のうち少なくとも1つは偶数である.}
\begin{proof}
    対偶は「$2 \nmid a,b,c \Rightarrow a^2 + b^2 \neq c^2$」 \\
    $a = 2k + 1, b = 2l + 1, c = 2m + 1$とおくと
    \begin{align*}
        a^2 + b^2 &= (2k + 1)^2 + (2l + 1)^2 \\
                  &= 4k^2 + 4l^2 + 4k + 4l + 2 \\
                  &= 2(2k^2 + 2l^2 + 2k + 2l + 1) \\
        c^2 &= (2m + 1)^2 = 4m^2 + 4m + 1 = 2(2m^2 + 2m) + 1
    \end{align*}
    よって$a^2 + b^2$と$c^2$の偶奇が一致しないから$a^2 + b^2 \neq c^2$
\end{proof}

\end{document}
